\chapter*{Preface}

\section*{The Need of Integration}

There are hundreds of books about software engineering, softwawre
development, programming, software testing ...  There are also many
books on artificial intelligence, machine learning, and related
topics.  Why do we write this book?  Why should you spend time reading
this book?  How is this book different from any other book?  We write
this book because {\it we think there is a need.}  {\it Existing books
  are independent of each other.  Instead, this book resides at the
  intersection of three topics: (1) software engineering, (2) machine
  learning, and (3) programming.  }


Few software engineering books show programs. Some machine learning
books do show programs that implementing the algorithms but never
explain how to manage the software.  Few programming books talk about
machine learning.  As these fields become increasingly important and
deeply intertwined, a book that integrates these topics becomes
essential.

This is that book.

Why is it necessary to have a book that covers all three topics
(software engineering, programming, and machine learning)
simultaneously?  Existing university education is strucuted into
majors: accounting, biology, computer science, electrical engineering,
literature, mathematics, music, statistics, theatre, and so on.  Each
major has a curriculum and students take sequences of courses building
their expertise in the particular field. Each course focuses on a
specific topic.  It is expected that a student would become an expert
in the field after taking a series of courses.  This approach,
however, often lead to the situations where students ``can see trees but
overlook forests'' because they cannot ``connect the dots''.  This is
understandable because each course focuses on depth (i.e., ``trees'')
without teaching how to see forests.  This traditional approach is
no longer valid becasue rapid technology progresses make integration
necessary.
Multidisciplinary projects are everywhere. Engineers needs to be aware
of the social impacts, as well as legal booundaries, of technologies.

Is there a need for a book, any book, on any subject since almost
everything is now available on the Internet?  Any book becomes
obsolete at the moment when it is published.  The difference between
books and the Internet is that (good) books can provide systematic
approaches to learning. The Internet is excellent for looking up
answers of specific questions. However, in many cases, asking the
right questions requires deep understanding of the topics.  This
creates a circular problem: Without deep understanding, it is
difficult to ask meaningful questions and find useful answers on the
Internet. Without useful answers, it is difficult to learn and acquire
deep understanding to ask the right questions.

\section*{Principles in Writing this Book}

Research shows that ``learning from mistakes'' is essential in
understanding and mastering any concept or skill.  However, few books
talk about common mistakes, how to recognize, correct, or prevent
mistakes.  Most books follow the same flow: describing a problem
followed by a solution.  As a result, most readers do not recognize
mistakes and do not know how to prevent mistakes.  This book is
different from most books.  We set a few principles when writing this
book:

\begin{itemize}
\item This book includes many common mistakes, why they are wrong, how
  to detect and prevent them.
  
\item This book emphasizes actions. Readers are asked to {\it do
  things} because doing is the best way to learn.

\item This book is not a dictionary nor an encyclopedia.  It is
  intended to provide readers with actionable knowledge so that they
  can learn enough basic skills and look up additional information on
  the Internet.  

\item This book provides good ways to solve common problems.
  The book does not intend to explain all possible ways
  solving all possible problems. Many useful solutions
  are available online; the problem is to have 
  the right skills of asking the right questions for
  finding the information.

\item This book aims to provide intuitive explanation of the concepts
  so that readers can understand the concepts.  When a choice must be
  made between ease of understanding and efficiency, this book selects
  the former (ease of understanding).  Thus, readers are advised that
  many solutions presented in this book can be improved and become
  more efficient.
  
\end{itemize}

\section*{Audience and Expected Background}

This book is written for sophomore students in science, technology,
engineering, or mathematics (STEM), assuming that they know algebra
and calculus. Readers should have already solved some problems using
computer programs. This book does not explain how to install software
packages because there already many books on those topics.
This book uses free tools. Everything mentioned in this book is free
of charge.

\section*{Open-Source Book}

We decide to make this book open-source: readers can see the Latex
source making this book.  Doing so allows us to release the book as it
is being written.  Please be advise that this book is work-in-progress
and mistakes are possible (in fact, likely).




