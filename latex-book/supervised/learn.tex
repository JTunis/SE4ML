\chapter{Supervised Learning}

Unsupervised learning means discovering patterns from data, just data.
Supervised learning means associating data with {\it labels}. When a
father holds a baby saying ``Daddy'', this is supervised learning: The
baby learns to associate the father's face (data) with the sound of
``Daddy'' (label). When a user identifies an email as spam and clicks
the ``Spam'' button in a mail reader, this is supervised learning: The
mail reader learns to associate the email (the data, may include the
subject or the content or the sender, or all of them) with the
property spam (label).

\index{supervised learning!classification}
\index{supervised learning!regression}

Supervised learning can be further divided into two types of problems:
{\it classification} and {\it regression}:

For classification problems, the labels (i.e., outputs) are discrete
values. They can be words, numbers, a finite set of options, etc.  The
labels are discrete because it makes no sense to have a label
``between'' two labels.  For example, the input of the voice of a
person and the output is the gender, either female or male.  It makes
no sense to say
\begin{equation}
  \frac{\text{female} + \text{male}}{2}.
\end{equation}
  That's why the labels are discrete.




