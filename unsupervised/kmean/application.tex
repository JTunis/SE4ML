\section{Application K-Mean}

The {\it k-mean} algorithm seems relatively simple. Is it actually useful?
Yes. Here are a few examples.

Imagine that you own several pizza stores and want to open new stores.
You want to decide where the new stores' locations.  You have the
budget to open at most five more stores (in this case $k$ is between 1
to 5).  You want to use the addresses of deliveries to determine the
stores' locations using the centroids of the clusters based on
customers' addresses.  This can be applied to many other scenarios:
For example, a bank wants to decide where to install ATM machines
based on customers' home and office addresses.  Another example: a
city's department of transportation wants to deploy traffic cameras to
monitor congestion based on the locations of past accidents.

\index{ATM: automatic teller machine}

It is common that the data needs to be ``cleaned'' before sending to a
clustering program.  For example, if you want to decide the locations
of pizza stores, you may want to exclude the customers that are too
far away from the other customers. Maybe you want to exclude the
locations that are, say, more than 20 km away from all the other
customers.  Maybe you also want to consider how often customers order
deliveries and give frequent customers larger weights so that the
stores are closer to these customers.  The same thinking can be
applied to selecting ATM locations: maybe you want to give the
customers that have high account balances. 




% Chapter~\ref{ch:clusterlimit} will explain some other factors for consideration.
