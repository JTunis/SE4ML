
\section{Application K-Mean}

The {\it k-mean} algorithm seems relatively simple. Is it actually useful?
Yes. Here are a few examples.

Imagine that you own several pizza stores and want to open new stores.
You want to decide the new stores' locations.  You have the budget to
open at most five more stores (in this case $k$ is between 1 to 5).
You want to use the addresses of deliveries to determine the stores'
locations using the centroids of the clusters based on customers'
addresses.  This can be applied to many other scenarios: For example,
a bank wants to decide where to install ATM machines based on
customers' home and office addresses.  Another example: a city's
department of transportation wants to deploy traffic cameras to
monitor congestion based on the locations of past accidents.

\index{ATM: automatic teller machine}

It is common that the data needs to be ``cleaned'' before sending to a
clustering program.  For example, if you want to decide the locations
of pizza stores, you may want to exclude the customers that are too
far away from the other customers. Maybe you want to exclude the
locations that are, say, more than 10 km away from all the other
customers.  Maybe you also want to consider how often customers order
deliveries and give frequent customers larger weights so that the
stores are closer to these customers.  The same thinking can be
applied to selecting ATM locations: maybe you want to give higher
weights to the customers that have high account balances or use ATM
more often.

Such ``data cleaning'' usually requires human judgement. In the
example of deciding pizza stores' locations, should the customers be
excluded if they are more than 10km away from all the other customers?
Why should it be 10km? Why not 5, or 8, or 15km?  If you want to
include the customers that order deliveries often even though they
live more than 10km away, how do you define ``often''?  Is more than
once per week considered often? Or more than twice per week?  Is one
week too short? Instead, you prefer to consider ordering more than
five times per month?  Using the {\it k-mean} method, or any machine
learning method, is not as simple as throwing data into a program and
get meaningful results effortlessly.  Doing meaningful data cleaning
often requires knowledge about the business (called {\it domain
  knowledge} by some people). You need to know how much it costs to
deliver pizza for customers living at different distances from the
store. You also need to know the traffic conditions: delivering pizza
to a customer living in the center of a city may take longer even
though the distance is shorter.

\index{domain knowledge}


% Chapter~\ref{ch:clusterlimit} will explain some other factors for consideration.
