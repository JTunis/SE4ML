\section{Implementation}

\index{binary tree}
\index{binary tree!leaf}

To implement hierarchical clustering, we will use two types of data
structures: {\it binary tree node} and {\it list node}.  Each binary
tree node represents a cluster. The original data points are stored in
the {\it leaf} nodes and each data point is a cluster of its own These
binary tree nodes are stored in a list.  In a binary tree, a node is a
leaf if it has no child.  Then, the cloest two clusters are fused
together into a single cluster.  The two clusters are removed from the
list and the new cluster is added to the list.  In each step, two
clusters are removed and one cluster is added.  As a result, the
number of clusters is reduced by one in each step.  This process
continues until only one cluster is left.

The program's starting point is relatively simple (as usual).  It
accepts one argument as the input file. Please notice that it is
different from the {\it k-mean} solution since hierarchical clustering
does not need the value of $k$.

\resetlinenumber[1]
\linenumbers
\begin{tt}
  \lstinputlisting{\progpath/unsupervised/hierarchical/solution/main.py}
\end{tt}
\nolinenumbers

The {\tt cluster} function creates a list, called {\tt HCList}, adds
the data points to the list, and clusters the data.

\resetlinenumber[1]
\linenumbers
\begin{tt}
  \lstinputlisting{\progpath/unsupervised/hierarchical/solution/cluster.py}
\end{tt}
\nolinenumbers

The code for {\tt TreeNode} is relatively simple. It stores
left and right children and can print.

\resetlinenumber[1]
\linenumbers
\begin{tt}
  \lstinputlisting{\progpath/unsupervised/hierarchical/solution/hctree.py}
\end{tt}
\nolinenumbers

