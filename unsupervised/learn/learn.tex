\chapter{Machine Learning}

Learning Objectives

\begin{itemize}
\item Understand the strength and limitations of machine learning
\item Distinguish  unsupervised learning,  supervised learning, and reinforcement learning
\end{itemize}

\section{Decision Problems}

\index{decision problem}

\subsection{Bring an Umbrella?}

Machine learning means using computer programs to discover patterns
(i.e., ``learn'') from data. How is this different from typical
computer programs?  Let us consider a {\it decision problem}:
Should you bring an umbrella or not?  This should be easy, right?


\begin{verbatim}
If it is raining then
   Bring an umbrella
Otherwise
   Do not bring an umbrella

% YHL: I have problems using 
% \usepackage{algorithmic} or
% \usepackage[linesnumbered,ruled,vlined]{algorithm2e}
\end{verbatim}

This is simple enough.  If you always follow this rule, you will soon
encounter situations when this rule is not good.  Maybe it is not raining
right now but it is expected to rain soon. Thus, you add one more rule:

\begin{verbatim}
If it is raining then
   Bring an umbrella
Otherwise
   If it is expected to rain soon then
      Bring an umbrella
   Otherwise
      Do not bring an umbrella
\end{verbatim}

If you follow these rules, you will soon discover that they are not
sufficient.  If you are going to walk in covered areas (such as subway
stations or shopping malls), maybe you do not need an umbrella.  Thus,
you add another rule:

\begin{verbatim}
If it is raining and walk outside covered areas then
   Bring an umbrella
Otherwise
   If it is expected to rain soon and walk outside covered areas then
      Bring an umbrella
   Otherwise
      Do not bring an umbrella
\end{verbatim}

If you are carrying a box and it does not rain heavily, maybe you
do not want to take an umbrella because it is inconvenient to carry
the box and to hold an umbrella at the same time. As a result,
the rules become

\begin{verbatim}
If it is raining and walk outside covered areas then
   If carrying a box and the rain is light then
       Do not bring an umbrella
   Otherwise
       Bring an umbrella
Otherwise
   If it is expected to rain soon and walk outside covered areas then
      Bring an umbrella
   Otherwise
      Do not bring an umbrella
\end{verbatim}

The rules become more and more complex.  You may also want to consider
whether it is windy. If it is, you may want to wear a raincoat instead
of bringing an umbrella. If you want to ride a bike, you may want to
choose a rain coat instead of an umbrella.  As you can see, this
decision problem has so many different scenarios hat writing these if
conditions become really complex.

\subsection{Cross a Stree Intersection?}

If you drive a car, should you cross a street intersection or not?
You may think this is easy:

\begin{verbatim}
If the traffic light is green then
   Cross the intersection
Otherwise
   Do not cross the intersection
\end{verbatim}

Sometimes, you cannot enter the inserction because it is blocked
by vehicles already. If you enter the intersection, you will park
at the intersection and worsen traffic jam. Thus, you add more rules:

\begin{verbatim}
If the traffic light is green and the intersection is not occupied then
   Cross the intersection
Otherwise
   Do not cross the intersection
\end{verbatim}

If you hear the siren of an ambulance, you should allow it to pass first.
Consequently, the rule gets more complex:

\begin{verbatim}
If the traffic light is green and the intersection is not occupied and no siren then
   Cross the intersection
Otherwise
   Do not cross the intersection
\end{verbatim}

If there is a jaywalker, you don't want to hit the person. Let's add that into the rule:

\begin{verbatim}
If the traffic light is green and the intersection is not occupied and no siren and no jaywalker then
   Cross the intersection
Otherwise
   Do not cross the intersection
\end{verbatim}

If there is a construction and a flagman, you should follow the
flagman's instruction, not the traffic light.  

\begin{verbatim}
If a flagman signals to cross the intersection then
   Cross the intersection
Otherwise
   If the traffic light is green and the intersection is not occupied and no siren and no jaywalker then
       Cross the intersection
   Otherwise
       Do not cross the intersection
\end{verbatim}

We can keep adding more and more rules to cover many different
scenarios.  As you can see, this decision problem may consider many
factors and writing down these rules become increasingly complex and
difficult.

\subsection{Approve a Mortgate Application?}

If you are a bank manager and evaluate a mortgage application, how do
you decide whether to approve or not?  If you approve the application
and the person pays back regularly, your bank makes money from fees
and interests.  If the person fails to pay (called ``defaults''), the
bank {\it may} lose money. It is possible that the bank does not lose
money if the house's value is sufficient to cover the mortgage
(through foreclosure). Foreclosure can be a lengthy process and most
banks want to avoid it.  How do you decide? Maybe you decide
based on the person's regular income:


\begin{verbatim}
If the montly income is great than twice or the monthly payment then
   Approve the mortgate application
Otherwise
   Deny the mortgate application
\end{verbatim}

If you do this, your bank will likely lose a lot of money because you
need to consider other factors, for example, whether the person
already has a lot of debt. You probably also want to consider whether
this person has a record of failing to pay bills.  If the house is not
in a popular location, you definitely want to avoid the possibility of
foreclosure. If the person has been doing business with the bank for
several years, you would trust this person more than a new customer.
Do you want to consider whether this person already own one or more
houses?  How about this person's age?  Would the marital status affect
your decision? If you want to consider many factors, are some factors
are more important than the others? How do you choose the important
factors?  This problem, again, shows that considering many factors and
writing down the rules become increasingly complex and difficult.

\section{Knowledge in Data}
