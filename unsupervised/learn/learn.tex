\newcommand{\learnpath}{\thischapterpath}

\chapter{Machine Learning}

\section{Learning Objectives}


\begin{itemize}
\item Learn what machine learning is 
  
\item Understand the strength and limitations 
  
\item Distinguish  unsupervised learning,  supervised learning,  reinforcement learning, and transfer
  learning
\end{itemize}

\section{Decision Problems}

\index{decision problem}


\subsection{Bring an Umbrella or not?}

Machine learning means using computer programs to discover patterns
(i.e., ``learn'') from data.  \marginnote{Strictly speaking, data is
  plural and datum is singula.  However, most people treat data as
  singular now and this book follows the same convention treating data
  as singular.}  How is this different from typical computer programs?
Let us consider a {\it decision problem}: Should you bring an umbrella
or not?  This should be easy, right?

\begin{algorithm}
%    \caption[]{Decision to Bring An Umbrella (1)}
    \begin{algorithmic}[1]
      \If {It is raining}
      \State Bring an umbrella
      \Else
      \State Do not bring an umbrella
      \EndIf
    \end{algorithmic}
\end{algorithm}

This is simple enough.  If you always follow this rule, you will soon
encounter situations when this rule is not good.  Maybe it is not raining
right now but it is expected to rain soon. Thus, you add one more rule:

\begin{algorithm}
%    \caption[]{Decision to Bring An Umbrella (2)}
    \begin{algorithmic}[1]
      \If {(It is raining) or
        (It is expected to rain soon)}
      \State Bring an umbrella
      \Else
      \State Do not bring an umbrella
      \EndIf
    \end{algorithmic}
\end{algorithm}

If you follow these rules, you will soon discover that they are still
insufficient.  If you are going to walk in covered areas (such as
subway stations or shopping malls), maybe you do not need an umbrella.
Thus, you add another rule:

\begin{algorithm}
%    \caption[]{Decision to Bring An Umbrella (2)}
    \begin{algorithmic}[1]
      \If {(You will walk in uncovered areas) and ((It is raining) or
        (It is expected to rain soon))}
      \State Bring an umbrella
      \Else
      \State Do not bring an umbrella
      \EndIf
    \end{algorithmic}
\end{algorithm}


If you are carrying a box and it does not rain heavily, maybe you
do not want to take an umbrella because it is inconvenient to carry
the box and to hold an umbrella at the same time. As a result,
the rules become

\begin{algorithm}
%    \caption[]{Decision to Bring An Umbrella (2)}
    \begin{algorithmic}[1]
      \If {(You will walk in uncovered areas) and (You are not carrying a box) and
        ((It is raining heavily) or
        (It is expected to rain heavily soon))}
      \State Bring an umbrella
      \Else
      \State Do not bring an umbrella
      \EndIf
    \end{algorithmic}
\end{algorithm}

The rules become more and more complex.  You may also want to consider
whether it is windy. If it is, you may want to wear a raincoat instead
of bringing an umbrella. If you want to ride a bike, you may want to
choose a rain coat instead of an umbrella.  Maybe it is very hot and
you want to walk in rain to cool.  As you can see, this decision
problem has so many different scenarios and writing all these {\tt
  if}-{\tt else} conditions becomes really complex.

\subsection{Cross a Stree Intersection?}

If you drive a car, should you cross a street intersection or not?
You may think this is really easy:

\begin{algorithm}[H]
    \begin{algorithmic}[1]
      \If {Traffic light is green}
      \State Cross the intersection
      \Else
      \State Do not cross the intersection
      \EndIf
    \end{algorithmic}
\end{algorithm}

Sometimes, you cannot enter the inserction because it is blocked
by vehicles already. If you enter the intersection, you will park
at the intersection and worsen traffic jam. Thus, you add another rule:

\begin{algorithm}[H]
    \begin{algorithmic}[1]
      \If {(Traffic light is green) and (The intersection is clear)}
      \State Cross the intersection
      \Else
      \State Do not cross the intersection
      \EndIf
    \end{algorithmic}
\end{algorithm}

If you hear the siren of an ambulance, you should allow it to pass first.
Consequently, the rule gets more complex:

\begin{algorithm}[H]
    \begin{algorithmic}[1]
      \If {(Traffic light is green) and (The intersection is clear)
      and (There is no siren)}
      \State Cross the intersection
      \Else
      \State Do not cross the intersection
      \EndIf
    \end{algorithmic}
\end{algorithm}

If there is a jaywalker, you don't want to hit the person. Let's add that into the rule:

\begin{algorithm}[H]
    \begin{algorithmic}[1]
      \If {(Traffic light is green) and (The intersection is clear)
      and (There is no siren) and (There is no jaywalker)}
      \State Cross the intersection
      \Else
      \State Do not cross the intersection
      \EndIf
    \end{algorithmic}
\end{algorithm}

If there is a construction and a flagman, you should follow the
flagman's instruction, not the traffic light.  

\begin{algorithm}[H]
    \begin{algorithmic}[1]
      \If {There is a flagman}
      \State Follow the instruction
      \Else
      \If {(Traffic light is green) and (The intersection is clear)
      and (There is no siren) and (There is no jaywalker)}
      \State Cross the intersection
      \Else
      \State Do not cross the intersection
      \EndIf
      \EndIf
    \end{algorithmic}
\end{algorithm}

We can keep adding more and more rules to cover many different
scenarios.  As you can see, this decision problem may consider many
factors and writing down these rules become increasingly complex and
difficult.

\index{mortgage}
\subsection{Approve a Mortgate Application?}

If you are a bank manager and evaluate mortgage applications, how do
you decide whether to approve or not?  If you approve the application
and the person pays regularly, your bank makes money from fees and
interests.  If the person fails to pay (called ``defaults''), the bank
may lose money. It is possible that the bank does not lose money
if the house's value is sufficient to cover the mortgage (through
foreclosure). Foreclosure can be a lengthy process and most banks want
to avoid it.  How do you decide? Maybe you decide based on the
person's regular income: \index{default} \index{foreclosure}


\begin{algorithm}[H]
    \begin{algorithmic}[1]
      \If {An applicant's monthly income is more than twice the mortgage payment}
      \State Approve the application
      \Else
      \State Deny the application
      \EndIf
    \end{algorithmic}
\end{algorithm}

If you do this, your bank will likely lose a lot of money because you
need to consider other factors, for example, whether the person
already has a lot of debt. You probably also want to consider whether
this person has a record of failing to pay bills.  If the house is not
in a popular location, you definitely want to avoid the possibility of
foreclosure. If the person has been doing business with the bank for
several years, you would trust this person more than a new customer.
Do you want to consider whether this person already owns one or more
houses?  Should you consider this person's age?  Would the marital
status affect your decision?  You may want to consider the economy as
well. If economy is strong, this applicant is likely to keep the
current job or even get a raise; thus, this applicant is likely to pay
the mortgage. If economy is weak, this applicant may lose the current
job and fails to pay.  If you want to consider many factors, are some
factors are more important than the others? How do you choose the
important factors?  This problem, again, shows that considering many
factors and writing down the rules become increasingly complex and
difficult.

\section{Decisions and Feedback}

In our everyday life, we make hundreds of decisions: what clothes to
wear, where to go for lunch, what to buy in a store, what birthday
gift to send to a friend, etc.  The previous examples show that
decision problems often need to consider many factors.  The reason we
need to consider many factors is to prevent making wrong decisions.
What are ``wrong'' decisions? If you bring an umbrella and it does not
rain, it is a wrong decision.  If you do not bring an umbrella and it
rain heavily, it is a wrong decision.  Some wrong decisions have
negligible consequences: bringing an umbrella (if it is small and
light) without using it may not be a big problem.  Some wrong
decisions can have dire consequences: approving a mortgage and it
defaults, the bank loses a lot of money.

What is a ``wrong'' decision? It may not be so obvious.  A bank may
deny all mortgage applications that have slight chances of
defaults. This may completely avoid defaults but the bank also loses
opportunities making money from the applications that may, but do not,
default.  In this case, ``preventing defaults'' and ``making money''
are two related but different goals.

\index{feedback}

After knowing the consequences of a decision, we may conclude that it
is a right or a wrong decision. This is the {\it feedback} of the
decision.  With the feedback, we hope to make better decisions in the
future. For example, if the bank approves a mortgage and it defaults,
the bank would probably deny the next application that is ``similar''
to the defaulted one.  The problem is how to determine two mortgage
applications are ``similar''.

For some problems, it is impossible (or almost impossible) to know
whether a wrong mistake has been made. Suppose you are a bank manager
and you deny a mortgage application. You will not know whether this
person would be able to pay mortgage regularly because you have denied
the application.


\index{feedback}

\section{Knowledge in Data}

\index{infer}
\index{pattern}

\marginnote{Many important decisions are driven by data. A famous
  example is the 2010 study by Carmen Reinhart and Ken Rogoff about
  the relationships between debts and economic growth.  Their study
  has significant impacts on the austerity policies adopted by some
  countries after the 2008 financial crisis.  Readers are encouraged
  to read about the study and many discussions about the validity of
  the study.  }


The examples described earlier are decision problems: whether to bring
an umbrella, whether to enter a traffic intersection, whether to
approve a mortgage application, etc. Each problem needs to consider
many factors and it is not always clear which factor is more important
than the other. One way to solve these problems is to examine similar
scenarios in the past and their results: if the current mortgage
application is ``similar'' to one that was approved in the past, was
that approval a right decisoin (i.e., did not default)?  Instead of
writing rules, this new approach uses past data to guide future
decisions.  This is what {\it machine learning} can be helpful:
computer programs discover (i.e., ``learn'') patterns from data.  Past
data may help decide which factors are important for making decisions
{\it if the past data and the new decision problem have similar
  patterns.}  Having similar patterns is an essential assumption in
machine learning.  Think about how humans learn: a person observes
something and then when the person sees {\it similar} problems, the
person uses past knowledge and experience to {\it infer} the solutions
for the new problems.  If a person has never seen anything similar,
the person would not able to draw from past knowledge or experience.

Machine learning relies on the assumption that past observations and
new, unseen, situations have similar patterns.  This assumption is
essential to the success of machine learning.  Imagine that you are
the bank manager and have discovered a good way to determine whether
to approve or deny mortgage applications.  If you move to another city
or another country, your method may be wrong more often than you
expect. Maybe the demographics are different. Maybe the cultural norms
are different.  Maybe the real estate markets are different.  This
indicates that your machine learning method has its limitations.  You
may need to add some more data into building your knowledge about the
new problem.

\index{limit of machine learning}

\index{supervised learning}
\index{unsupervised learning}
\index{reinforcement learning}
\index{reward}


\section{Supervised, Unsupervised,  Reinforcement, and Transfer Learning}

There are different types of
learning~\cite{Goodfellow2016DeepLearning}.  {\it Supervised learning}
means that there is a ``teacher'' telling a ``student'' what is right
or wrong. Imagine that a teacher shows images of flowers and tell
students that these are flowers. The teacher shows another image of an
elephant and says that it is an elephant.  {\it Unsupervised learning}
has no teacher. Imagine that you want to stock your store on a Friday
evening for sales on Saturday.  There is no correct answer what
products you should put on shelves.  You can analyze the past sales
records, together with factors such as weather and season.  You may
also want to consider whether there is a major sport event on that
Saturday.  This is different from supervised learning because there is
no teacher telling you ``Yes, you should stock this item on shelves.''
or ``No, do not stock that item because few people will buy it this
coming Saturday.''  Unsupervised learning is often used to discover
(unknown) properties in data, for example, what people buy on a
Saturday.  The third type of learning is called {\it reinforcement
  learning}.  It considers sequences of actions (such as moves in
chess) and the {\it rewards} (such as winning a chess game) of these
actions.  Reinforcement learning is different from supervised learning
because most decisions cannot be consider right or wrong. Some
decisions such as checkmate are obviously right decisions but the
effects of most decisions are unkonwn until much later. Instead, the
sequence of decisions leads to a result, either winning or
losing. Reinforcement learning is usually used for developing
strategies solving problems through sequences of
actions~\cite{Sutton2017ReinforcementLearningIntroduction}.  The
fourth type of learning is called {\it transfer learning}.  The
knowledge learned from the sample data is ``transferred'' to a new set
of test data.  An analogy is that a person learns English and then
uses the knowelege about sentence structures and tenses to learn
French.
% Transfer learning can be useful if there are many examples in the
% first set of data and much fewer examples in the second set of data.
% Table~\ref{table:threetypesoflearning} summarizes the three types of
% learning.

\index{transfer learning}


Supervised learning may be the most familar form of learning: babies
learn parents' faces when the parents say ``Daddy'' and ``Mommy''.
Students learn from teachers in classrooms.  Supervised learning,
however, can be expensive because teachers are needed.  As computer
technologies improve, acquiring data becomes very easy and
inexpensive.  Spending \$100, you can buy a video camera and the it
can easily generate thousands of images (more precisely, video frames)
per day. Teachig computer the information in the images requires
humans as teachers because computer programs cannot perfectly analyze
images yet.  Teaching computers by marking what is in the images is
called {\it labeling} or {\it annnotating}.  In some cases, labeling
can be {\it crowdsourced}.  Labeling one million images by humans
would not be easy~\cite{Russakovsky2015ImageNetLargeScale211252}.
\index{crowdsourced} In some other cases, the ``teachers'' of
computers must have special qualifications; for example, medical
images are evaluated by trained medical doctors.

Unsupervised learning is applied when a person mimicks the behaviors
of another person.  Imagine that you have a vacation in Korea and hear
people saying ``안녕하세요'' when they meet.  Even though nobody
(i.e., there is no teacher) tells you what it means, you start saying
``안녕하세요'' when you meet people.  This is an example of
unsupervised learning.  Unsupervised learning can also be used to
discover patterns in data, for example, people that buy apples are
liekly to buy organes also.  Web search engines are examples of
unsupervised learning.  These engines analyze many websites and rank
websites for different search keywords. There is no ``teacher''
specifying the correct orders.

\index{web search engine}

\index{visual data}
\index{label} \index{annotating}


\begin{comment}
\vspace{0.1in}
\begin{table}
  \caption{Four types of learning}
  \begin{tabular}{p{1in}p{1in}p{1in}p{1in}p{1in}}
    &    {\bf Supervised} & {\bf Unsupervised} & {\bf Reinforcement} & {\bf Transfer}\\
    \hline
    Teacher & Yes & No & No \\
    Correct Answer & Yes & No & No \\
    Consider Sequences & No & No & Yes \\
    Applications & Answer Yes/No & Cluster data & Develop strategies \\
  \end{tabular}

  \label{table:threetypesoflearning}
\end{table}
\vspace{0.1in}

\end{comment}

\begin{comment}
http://incompleteideas.net/book/bookdraft2017nov5.pdf
Reinforcement Learning: An Introduction
Richard S. Sutton and Andrew G. Barto
\end{comment}



\index{learning}

\section{Define Learning}

  We have talked about ``learning'' without actually defining it.
  What is learning? Michelle~\cite{Mitchell1997MachineLearning}'s
  definition is

  
  
  {\it
    A computer program is said to learn from experience E with respect
    to some class of tasks T and performance measure P, if its
    performance at tasks in T, as measured by P, improves with
    experience E.}

  

  To explain this in a more intuitive way, a computer program can
  learn if it can ``get better'' by doing something more.  One way to
  understand learning is by comparing it with something that cannot
  learn. Consider the calculator program on your mobile phone. It does
  not get better after you use it.  In contrast, a program that
  determines whether an email is spam may get better after you mark
  some emails as spam~\cite{Hastie2009ElementsStatisticalLearning}.
  By marking spam emails, you play the role of a teacher and this is
  an example of supervised learning.




This definition does not speficy what is ``experience''.  From
computers' viewpoint, the experience often refers to ``data''.  If
more data is used (assuming the data follows specific patterns), then
the computer program can perform better (such as making more correct
decisions in mortgage applications).

What is {\it machine learning} really?  Machine learning is pretty
broad (and somewhat vague).  In this book (and many other books),
machine learning refers to {\it statistical learning} or {\it
  data-driven discovery}: finding information from data.  Successful
machine learning often requires vast amounts of data to learn from.
Machine learning discovers {\it patterns} in the data and uses the
patterns to {\it predict} or {\it infer} that unseen data has the same
(or similar) patterns.  For example, a computer program may discover
that a person has a high debt-income ratio is likely to default in a
mortgage.  If a future mortgage applicant has a high debt-income
ratio, the program could suggest denying the application due to the
higher risk.

As explained earlier, machine learning can be used when many factors
need to be considered.  Machine learning has already been used in many
applications~\cite{Alpaydin2010IntroductiontoMachine}, such as
improving customer relationships, making financial decisions,
diagnosize illness, identify spam emails, recognize speeches and
objects in images.


\section{Limitations of Machine Learning}

Machine learning is not perfect; machine learning has some
limitations.  First, what can be learned depends on the input data. If
some important pieces of data are missing (for example, there are no
cases of high debt-income ratios), then the computer program cannot
learn.  To think of this in a different way, a person that grows up
inland and has never seen a cargo ship will thus not know existence of
cargo ships.  Second, it is difficult to determine when the data is
``sufficient'' or ``representative''.  The patterns are unknown
(otherwise, there is no need to learn) so it is hard to tell when
there is enough data to discover the patterns.  Third, the data may be
``biased'' and the it is not easy to define success. Imagine that you
are designing a machine learning program to diagnose a rare illness.
If a person has this illness, the program says ``Yes''; otherwise, the
program says, ``No''.  Suppose the probability of this illness is one
out of 100,000 people.  The program would be 99.999\% accurate if it
always says ``No''.  However, this high accuracy does not really help.


\begin{comment}
Fourth, each machine learning program reflects a specific {\it model}
that is designed to recognize the patterns in the data.  Different
models have different capabilities: some models can recognize complex
patterns and some others cannot.  This is an analogy about models'
capabilities: humans are the only known species that can handle
written communications.  Some other animals can communicate in sound
that is not understood by humans. It is not yet clear whether humans
will ever be able to understand the sound.  Maybe human brains are
incapable of learning the sound.

\end{comment}

\index{black swan event}

\section{Black Swan Event}

    
  A {\it black swan event} is something that has never been seen
  before and thus considered impossible~\cite{Taleb2010BlackSwan}.
  People used to believe that swans must be white. Apparently, it is
  not possible to learn by looking at white swans and infer the
  existence of a black swan.

  
  Black swan events are everywhere, if you pay close attention.
  Before 1969/07/20, nobody could expect that a man would be able to
  walk on the moon. Before April 2010, nobody would expect a volcano
  eruption could cause worldwide disruption of air travel.  Before the
  first iPhone was announced, there was no iPhone. Before Michael
  Phelps won 28 Olympic medals, nobody had won 28 Olympic medals.


  If something has been observed, it is definitely possible.  After a
  black swan has been seen, people know that swans can be black.  If
  something has not been observed, it is difficult to say whether it
  is impossible or not. Maybe it can be observed later.

  Even though black swan events cannot be inferred  from known patterns,
  it is possible to predict the occurrence {\it by some people} that
  are willing to challenge these patterns and consider
  possibilities that have no been observed (i.e., learned). This
  is where {\it imagination} and {\it creativity} come in.
  This book focuses on learning  and does not discuss imagination or
  creativity further.

  The following chapters will describe some widely used machine
  learning techniques for analyzing data.

  \index{imagination}
  \index{creativity}
    
  
\index{statistical learning}
\index{data-driven discovery}
\index{infer}
\index{predict}
\index{data bias}
\index{model}

\begin{comment}

\section{Applications of Machine Learning}


\vspace{0.1in}

\index{computer vision}

    {\bf Computer Vision (not sure it belongs here, maybe move to
    somewhere later}


  Computer vision is the technologies understanding what is in

 The
technologies intend to solve many different types of problems, for
example,

 Classification of the content: Is the visual data taken indoor
  or outdoor?  Is this a sport event?  Is this downtown or forest?

 Object Detection: Does the visual data contain a car, a person,
  a bridge, a computer, or something else?  Where is the detected
  object?

 Action: Is a person playing tennis? Are two people talking?
  Is that airplane taking off?

 Emotion: Is a peron happy, sad, bored?





\end{comment}
