\chapter*{Preface}

\section*{Why Is This Book Needed?}

There are hundreds of books about software engineering, softwawre
development, programming, software testing ...  There are also many
books on artificial intelligence, machine learning, and related
topics. Many books and movies decribe future worlds where machines are
capable of learning, reasoning, and making decisions.
Why do we write this book?  Why should you spend time reading
this book?  How is this book different from any other book?  Like many
authors, we want to write this book because we perceive a need for it.
{\it Because we think there is a need for this book.}


\begin{tabular}{|p{4in}|}
  {\it The problem is that there are too many books on these topics.
    These books are independent of each other.  Instead, this book
    resides at the intersection of three topics: software engineering,
    machine learning, and programming.  }
\end{tabular}  

Few Software Engineering books show programs. Some Machine Learning
books show programs that implementing algorithms but never explain how
to manage the software.  Very few programming books talk about
applications of machine learning.  As these fields become increasingly
important and deeply intertwined, a book that integrates these topics
becomes essential.  This is that book.

Why is it necessary to have a book that covers all three topics
simultaneously?  Existing university education is strucuted into
majors: accounting, biology, computer science, electrical engineering,
literature, mathematics, statistics, theatre, and so on.  Each major
has a curriculum and students take sequences of courses building their
expertise in the particular field. Each course focuses on a specific
topic.  It is expected that a student would become an expert in the
field after taking a series of courses.  This approach, however, often
lead to the situations where people ``can see trees but overlook
forests'' because they cannot ``connect the dots''.  This is
understandable because each course focuses on depth (i.e., ``trees'')
and without teaching how to see forests.  This traditional approach is
no longer valid becasue rapid technology progresses make integration
necessary.  {\it Multidisciplinary} projects are everywhere.

Is there a need for a book, any book, on any subject since almost
everything is now available on the Internet?  Any book becomes
obsolete at the moment when it is published.  The difference between
books and the Internet is that books (good books) can provide
systematic approaches to learning. The Internet is excellent looking
up answers for specific questions. However, in many cases, asking the
right questions requires deep understanding of the topics.  This
creates a circular problem: Without deep understanding, it is
difficult to ask meaningful questions and find useful answers. Without
useful answers, it is difficult to learn and acquire deep
understanding to ask the right questions.

\section*{Principles in Writing this Book}

This book is different from most (maybe all) available books.
We set a few principles when writing this book:

\begin{itemize}
\item This book emphasizes actions. Readers are asked to {\it do
  things} because doing is the best way to learn.

\item This book eliminates as much redudant materials as
  possible.  There is no history about machine learning
  of software engineering; there are already many books
  on these subjects.

\item This book is written for ``advanced beginners'': they have
  already written some programs solving small (``toy'')
  problems. Also, this book does not explain how to manage your
  computer.  There are already hundreds of books about
  how to use computers and install software packages.

\item This book is not a dictionary nor an encyclopedia.
  It is intended to provide readers with actionable
  knowledge so that they can learn enough basic skills
  and look up additional information on the Internet.

\item This book provides good ways to solve common problems.
  The book does not intend to explain all possible ways
  solving all possible problems. Many useful solutions

\item This book uses free tools. Everything mentioned in
  this book is free of charge.

\item This book solves ``real'' problems.  Most books
  use simplified problems

\item This book aims to provide intuitive explanation of the concepts
  so that readers can understand the concepts.  When a choice must be
  made between ease of understanding and efficiency, this book selects
  the former.  Thus, readers are advised that 
  
\end{itemize}  



