\documentclass{tufte-book}
% \usepackage{algorithmicx}
\usepackage[noend]{algpseudocode}
\usepackage[]{algorithm}
\usepackage{amsmath}
\usepackage{amssymb}
\usepackage{dsfont}
\usepackage{enumitem}
\usepackage{fixltx2e,fix-cm}
\usepackage{graphicx}
\usepackage{hhline}
\usepackage{lineno}
\usepackage{listings}
\usepackage{longtable}
\usepackage{makeidx}
\usepackage{multicol}
\usepackage{multirow}
\usepackage{subfigure}
\usepackage{url}
\usepackage{verbatim}
\usepackage[none]{hyphenat}

% GKT adding these commands to support part and chapter path macros

\newcommand{\thispartpath}{}
\newcommand{\thischapterpath}{}
  
\makeindex

% only for draft
\usepackage{draftwatermark}
\SetWatermarkScale{0.2}
\SetWatermarkText{DRAFT. Work-in-Progress. \today}

\begin{comment}

\newtheorem{theorem}{Theorem}
\newtheorem{exercise}{Exercise}[chapter]
\newtheorem{example}{Example}
\newtheorem{definition}{Definition}
\newtheorem{proof}{Proof}


% make table of content clickable, suggested at
% https://tex.stackexchange.com/questions/73862/how-can-i-make-a-clickable-table-of-contents

\usepackage{color}
%May be necessary if you want to color links
\usepackage{hyperref}
\hypersetup{
  colorlinks=true,
  %set true if you want colored links
  linktoc=all,
  %set to all if you want both sections and subsections linked
  linkcolor=blue,
  %choose some color if you want links to stand out
}
\end{comment}

% I am not sure what these mean- YHL

\begin{comment}

% \lstset{breaklines=true}
% \lstset{breakatwhitespace=true}
\lstset{
  language=C,
  breakatwhitespace=true,
  breaklines=true,
  basicstyle=\ttfamily,
  keywordstyle=\bf,
  morekeywords={void}
}
\frenchspacing
\tolerance=5000
\end{comment}
