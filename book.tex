\input{settings}

\begin{comment}

to find all files that contain a term (useful for creating index)

grep -r term * | grep tex | sed 's/:/ /g' | awk '{print $1}' | sort | uniq

\end{comment}

\newcommand{\progpath}{\basepath/programs}

\makeatletter
\def\seealso#1#2{{\em see also\/} #1, #2}
\makeatother


\begin{document}

\begin{comment}
reduce space

titlesec
http://www.ctex.org/documents/packages/layout/titlesec.pdf
wrapfig

\end{comment}

\frontmatter
\title{
Machine Learning and Software Engineering: Build
Trustable Systems and Avoid Pitfalls
}
\author{Yung-Hsiang Lu
and George K. Thiruvathukal}
\maketitle

\begin{comment}
Topics:

Part1: Software Tools and Development Process
   Chapter 0: Python Basics
   Chapter 1: Version Control, Pull Requests
   Chapter 2: Continuous Integration, Unit Tests
   Chapter 3: Branch and Web Services
   Chapter 4: Documentation
   Chapter 5: Graphics Packages

Part 2: Machine Learning I: Unsupervised Learning (mathematics)
   Chapter 6: k-mean
   Chapter 7: Hierarchical Clustering
   Chapter 8: Support vector machines

Part 3: Machine Learning II: Supervised Learning (mathematics)
   Chapter 9: Linearc Regression
   Chapter 10: Logistic Regression
   Chapter 11: Neural Network

Part 4: Implementing Machine Learning
   Chapter 12: k-mean
   Chapter 13: Hierarchical Clustering
   Chapter 14: Support vector machines
   Chapter 15: Linearc Regression
   Chapter 16: Logistic Regression
   Chapter 17: Neural Network

Part 5: Applications
   Chapter 18: Movie Preference
   Chapter 19: Handwritten Digit Recognition
   Chapter 20: Object Detection in Images
   
     

Reinforcement Learning
   

\end{comment}

\tableofcontents
\listoffigures
\listoftables

\begin{comment}

\newcommand{\frontpath}{\basepath/frontmatter}
\chapter*{Foreword}



\begin{flushright}
{\it } 

\end{flushright}


 

\newpage
\chapter*{Preface}

\section*{Why Is This Book Needed?}

There are hundreds of books about software engineering, softwawre
development, programming, software testing ...  There are also many
books on artificial intelligence, machine learning, and related
topics. Many books and movies decribe future worlds where machines are
capable of learning, reasoning, and making decisions.
Why do we write this book?  Why should you spend time reading
this book?  How is this book different from any other book?  Like many
authors, we want to write this book because we perceive a need for it.
{\it Because we think there is a need for this book.}


\begin{tabular}{|p{4in}|}
  {\it The problem is that there are too many books on these topics.
    These books are independent of each other.  Instead, this book
    resides at the intersection of three topics: software engineering,
    machine learning, and programming.  }
\end{tabular}  

Few Software Engineering books show programs. Some Machine Learning
books show programs that implementing algorithms but never explain how
to manage the software.  Very few programming books talk about
applications of machine learning.  As these fields become increasingly
important and deeply intertwined, a book that integrates these topics
becomes essential.  This is that book.

Why is it necessary to have a book that covers all three topics
simultaneously?  Existing university education is strucuted into
majors: accounting, biology, computer science, electrical engineering,
literature, mathematics, statistics, theatre, and so on.  Each major
has a curriculum and students take sequences of courses building their
expertise in the particular field. Each course focuses on a specific
topic.  It is expected that a student would become an expert in the
field after taking a series of courses.  This approach, however, often
lead to the situations where people ``can see trees but overlook
forests'' because they cannot ``connect the dots''.  This is
understandable because each course focuses on depth (i.e., ``trees'')
and without teaching how to see forests.  This traditional approach is
no longer valid becasue rapid technology progresses make integration
necessary.  {\it Multidisciplinary} projects are everywhere.

Is there a need for a book, any book, on any subject since almost
everything is now available on the Internet?  Any book becomes
obsolete at the moment when it is published.  The difference between
books and the Internet is that books (good books) can provide
systematic approaches to learning. The Internet is excellent looking
up answers for specific questions. However, in many cases, asking the
right questions requires deep understanding of the topics.  This
creates a circular problem: Without deep understanding, it is
difficult to ask meaningful questions and find useful answers. Without
useful answers, it is difficult to learn and acquire deep
understanding to ask the right questions.

\section*{Principles in Writing this Book}

This book is different from most (maybe all) available books.
We set a few principles when writing this book:

\begin{itemize}
\item This book emphasizes actions. Readers are asked to {\it do
  things} because doing is the best way to learn.

\item This book eliminates as much redudant materials as
  possible.  There is no history about machine learning
  of software engineering; there are already many books
  on these subjects.

\item This book is written for ``advanced beginners'': they have
  already written some programs solving small (``toy'')
  problems. Also, this book does not explain how to manage your
  computer.  There are already hundreds of books about
  how to use computers and install software packages.

\item This book is not a dictionary nor an encyclopedia.
  It is intended to provide readers with actionable
  knowledge so that they can learn enough basic skills
  and look up additional information on the Internet.

\item This book provides good ways to solve common problems.
  The book does not intend to explain all possible ways
  solving all possible problems. Many useful solutions

\item This book uses free tools. Everything mentioned in
  this book is free of charge.

\item This book solves ``real'' problems.  Most books
  use simplified problems 
  
\end{itemize}  




\newpage
\chapter*{Author, Reviewers, and Artist}

\section*{Author}

\index{ACM|see{Association for Computing Machinery}}
\index{Association for Computing Machinery} Yung-Hsiang Lu is a
professor at the School of Electrical and Computer Engineering in
Purdue University, West Lafayette, Indiana, U.S.A.  He is an ACM
(Association for Computing Machinery) Distinguished Scientist and ACM
Distinguished Speaker. He published ``Intermediate C Programming''
(CRC Press, ISBN 9781498711630).  This book provides deep insight
about how to prevent, detect, and remove software defects (i.e.,
``bugs'').  Dr. Lu and three Purdue students founded a Perceive
Inc. This technology company uses video analytics to improve shopping
experience in stores. This company receives two Small Business
Innovation Research (SBIR-1 and SBIR-2) grants from the National
Science Foundation.  Dr. Lu received the Ph.D. degree from the
Department of Electrical Engineering in Stanford University,
California, U.S.A.  \\

George K. Thiruvathukal is a Professor of Computer Science at Loyola
University Chicago.  He has published two books about software
engineering.


\section*{Artist}


The book's cover is painted by Kyong Jo Yoon.  Yoon is a Korean artist
and often places heroic figures in natural settings.  He is an adviser
of the Korean Fine Arts Association and his work is on display in the
Ann Nathan Gallery in Chicago, Illinois, U.S.A.







% \input{frontmatter/rules}

\mainmatter
\input{environment/env}
%  \input{storage/storage}
%  \input{recursion/recursion}
% \input{structure/structure}
% 
\section{Application K-Mean}

The {\it k-mean} algorithm seems relatively simple. Is it actually useful?
Yes. Here are a few examples.

Imagine that you own several pizza stores and want to open new stores.
You want to decide the new stores' locations.  You have the budget to
open at most five more stores (in this case $k$ is between 1 to 5).
You want to use the addresses of deliveries to determine the stores'
locations using the centroids of the clusters based on customers'
addresses.  This can be applied to many other scenarios: For example,
a bank wants to decide where to install ATM machines based on
customers' home and office addresses.  Another example: a city's
department of transportation wants to deploy traffic cameras to
monitor congestion based on the locations of past accidents.

\index{ATM: automatic teller machine}

It is common that the data needs to be ``cleaned'' before sending to a
clustering program.  For example, if you want to decide the locations
of pizza stores, you may want to exclude the customers that are too
far away from the other customers. Maybe you want to exclude the
locations that are, say, more than 10 km away from all the other
customers.  Maybe you also want to consider how often customers order
deliveries and give frequent customers larger weights so that the
stores are closer to these customers.  The same thinking can be
applied to selecting ATM locations: maybe you want to give higher
weights to the customers that have high account balances or use ATM
more often.

Such ``data cleaning'' usually requires human judgement. In the
example of deciding pizza stores' locations, should the customers be
excluded if they are more than 10km away from all the other customers?
Why should it be 10km? Why not 5, or 8, or 15km?  If you want to
include the customers that order deliveries often even though they
live more than 10km away, how do you define ``often''?  Is more than
once per week considered often? Or more than twice per week?  Is one
week too short? Instead, you prefer to consider ordering more than
five times per month?  Using the {\it k-mean} method, or any machine
learning method, is not as simple as throwing data into a program and
get meaningful results effortlessly.  Doing meaningful data cleaning
often requires knowledge about the business (called {\it domain
  knowledge} by some people). You need to know how much it costs to
deliver pizza for customers living at different distances from the
store. You also need to know the traffic conditions: delivering pizza
to a customer living in the center of a city may take longer even
though the distance is shorter.

\index{domain knowledge}


% Chapter~\ref{ch:clusterlimit} will explain some other factors for consideration.


\appendix
\backmatter
% \input{thankyou}
\bibliographystyle{unsrt}
\bibliography{references}
\end{comment}
\printindex
% \input{backmatter/back}

\end{document}
